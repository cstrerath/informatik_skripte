\documentclass[a4paper,12pt,ngerman]{report}

\usepackage[utf8]{inputenc}
\usepackage[T1]{fontenc}
\usepackage[ngerman]{babel}
\usepackage[german=quotes]{csquotes}
\usepackage{microtype}
\usepackage{amsmath}
\usepackage{amssymb}
\usepackage{graphicx}
\usepackage{listings}
\usepackage{booktabs}
\usepackage{tabularx}
\usepackage{acronym}
\usepackage{glossaries}
\usepackage[onehalfspacing]{setspace}
\usepackage{titlesec}
\usepackage[backend=biber, autocite=inline, style=numeric, sorting=none]{biblatex}
\usepackage[hidelinks]{hyperref}


\addbibresource{bibliography.bib}

\title{[TITEL DER SKRIPTE]}
\author{[IHR NAME]}
\date{\today}

\begin{document}
	
	\pagenumbering{Alph}
	\maketitle
	\thispagestyle{empty}
	
	\pagenumbering{Roman}
	\tableofcontents
	
	\cleardoublepage
	\pagenumbering{arabic}
	\setcounter{page}{1}
	
	\chapter{[KAPITEL 1 TITEL]}
	Dieses Kapitel enthält Beispiele für Zitationen. 
	
	Das Buch "The TeXbook" von Donald E. Knuth ist ein Standardwerk für TeX-Anwender \autocite{knuth1984}.
	
	Leslie Lamport, der Erfinder von LaTeX, beschreibt in seinem Artikel die Grundlagen des Textsatzsystems \autocite{lamport1994}.
	
	\chapter{[KAPITEL 2 TITEL]}
	[INHALT KAPITEL 2]
	
	\chapter{[KAPITEL 3 TITEL]}
	[INHALT KAPITEL 3]
	
	\printbibliography
	
\end{document}
